% $Id$

\usepackage{html}
\usepackage{xspace}
% \usepackage{verbatim}  % better implementation of verbatim environment
\usepackage{alltt} % a verbatim-like environment that accepts commands
\usepackage{graphicx}
\usepackage{longtable} % handles long tables, stretching over multiple pages

%begin{latexonly}
%%%% I need the latexonly call here (even with the % signs, but
%%%% without the slash) as this
%%%% helps latex2html *not* see the stuff inbetween

% this code hacked from that of R Chandrasekhar from UWA
\newif\ifpdf
\ifx\pdfoutput\undefined
	\pdffalse    % we are not running pdfLaTeX
\else
	\pdfoutput=1 % we are running pdfLaTeX
	\pdftrue
\fi

% add the color package
\ifpdf
\usepackage[usenames,dvipsnames]{color}
\else
\usepackage[usenames,dvips]{color}
\fi

% add the hyperref package
\ifpdf
\usepackage[pdftex]{hyperref}
\else
\usepackage[hypertex]{hyperref}
\fi

% defines the colour for the background of code examples
\definecolor{LightGrey}{gray}{0.9}

\usepackage[grey,times]{quotchap}
\usepackage{ccaption}
\usepackage{fancyhdr}   

\ifpdf
	\DeclareGraphicsExtensions{.pdf}  % this command defined in graphicx
	\pdfcompresslevel=9  % 0: no compression, 9: highest compression
			     % or, set compress_level 9 in file pdftex.cfg
\else
	\DeclareGraphicsExtensions{.eps}
\fi

\setlength{\oddsidemargin}{-1in}   \setlength{\evensidemargin}{-1in}
\addtolength{\oddsidemargin}{25mm}\addtolength{\evensidemargin}{20mm}
\setlength{\marginparwidth}{40pt} \setlength{\marginparsep}{10pt}
\setlength{\topmargin}{-5mm}      \setlength{\headsep}{0.5in}
\setlength{\textheight}{227mm}    \setlength{\textwidth}{165mm}

\brokenpenalty=10000   % dunno what this does, maybe handy

% this stops one figure taking up a whole page and lets more text onto
% the one page when a figure exists
\renewcommand{\floatpagefraction}{0.8} %   Default = 0.5

% improved version of caption handling
\captionnamefont{\scshape}
\captionstyle{}
\makeatletter
\renewcommand{\fnum@figure}[1]{\quad\small\textsc{\figurename~\thefigure}:}
\renewcommand{\@makecaption}[2]{%
\vskip\abovecaptionskip
\sbox\@tempboxa{#1: #2}%
\ifdim \wd\@tempboxa >\hsize
  \def\baselinestretch{1}\@normalsize
  #1: #2\par
  \def\baselinestretch{1.5}\@normalsize
\else
  \global \@minipagefalse
  \hb@xt@\hsize{\hfil\box\@tempboxa\hfil}%
\fi
\vskip\belowcaptionskip}
\makeatother

\pagestyle{fancy}

%%%%% Fancyhdr stuff
% give the header a bit more room, otherwise LaTeX will spew on each page
\addtolength{\headheight}{2.5pt}
% define how headers are marked, for details, see fancyhdr docs
\renewcommand{\chaptermark}[1]{\markboth{#1}{}}
\renewcommand{\sectionmark}[1]{\markright{\thesection\ #1}}

% define where sections, chapters and pagenumbers are put
% see fancyhdr docs for details
% the \nouppercase stops book.cls making the contents, bibliography
% and index headers from being all in uppercase.
% The options used here are essentially that in Lamport's book, but
% with small caps for the headings.
\fancyhf{}
\fancyhead[LE,RO]{\nouppercase{\thepage}}
\fancyhead[LO]{\sc \nouppercase{\rightmark}}
\fancyhead[RE]{\sc \nouppercase{\leftmark}}

\bibliographystyle{apsrev}

% optional packages
\usepackage[square,comma,numbers,sort&compress]{natbib}
		% this is the natural sciences bibliography citation
		% style package.  The options here give citations in
		% the text as numbers in square brackets, separated by
		% commas, citations sorted and consecutive citations
		% compressed 
		% output example: [1,4,12-15]
		% should I make this optional and have unsrt as standard?

\usepackage[nottoc]{tocbibind}  
				% allows the table of contents, bibliography
				% and index to be added to the table of
				% contents if desired, the option used
				% here specifies that the table of
				% contents is not to be added.
				% tocbibind needs to be after natbib
				% otherwise bits of it get trampled.

\usepackage{amsmath,amsfonts,amssymb} % this is handy for mathematicians and physicists
			      % see http://www.ams.org/tex/amslatex.html

% \usepackage{showkeys} % this shows what labels you are using for cross
		      % references

% add the listings package to pretty print the code output
\usepackage{listings}
\begin{latexonly}
\lstdefinestyle{myC++}{%
%\lstset{%
language=C++,
showstringspaces=false,
basicstyle=\small\ttfamily,
commentstyle=\color[named]{BrickRed}\ttfamily,
keywordstyle=\color[named]{Purple}\ttfamily,
%identifierstyle=\color[named]{Blue}\ttfamily,
%functionstyle=\color[named]{Blue}\ttfamily,
%typestyle=\color[named]{ForestGreen}\ttfamily,
stringstyle=\color[named]{Tan}\ttfamily,%
morekeywords={,complex,}%
frame=none,%
backgroundcolor=\color{LightGrey}%
}

\lstdefinestyle{myMatlab}{%
%\lstset{%
language=Matlab,
showstringspaces=false,
basicstyle=\small\ttfamily,
commentstyle=\color[named]{BrickRed}\ttfamily,
keywordstyle=\color[named]{Purple}\ttfamily,
%identifierstyle=\color[named]{Blue}\ttfamily,
%functionstyle=\color[named]{Blue}\ttfamily,
%typestyle=\color[named]{ForestGreen}\ttfamily,
stringstyle=\color[named]{Tan}\ttfamily,%
frame=none,%
backgroundcolor=\color{LightGrey}%
}

\lstdefinestyle{myScilab}{%
%\lstset{%
language=Scilab,
showstringspaces=false,
basicstyle=\small\ttfamily,
commentstyle=\color[named]{BrickRed}\ttfamily,
keywordstyle=\color[named]{Purple}\ttfamily,
%identifierstyle=\color[named]{Blue}\ttfamily,
%functionstyle=\color[named]{Blue}\ttfamily,
%typestyle=\color[named]{ForestGreen}\ttfamily,
stringstyle=\color[named]{Tan}\ttfamily,%
frame=none,%
backgroundcolor=\color{LightGrey}%
}

\lstdefinestyle{myShell}{%
%\lstset{%
language=ksh,
showstringspaces=false,
basicstyle=\small\ttfamily,
commentstyle=\color[named]{Black}\ttfamily,
keywordstyle=\color[named]{Black}\ttfamily,
%identifierstyle=\color[named]{Blue}\ttfamily,
%functionstyle=\color[named]{Blue}\ttfamily,
%typestyle=\color[named]{ForestGreen}\ttfamily,
stringstyle=\color[named]{Black}\ttfamily,%
frame=none,%
backgroundcolor=\color{LightGrey}%
}

\lstdefinestyle{myPerl}{%
%\lstset{%
language=perl,
showstringspaces=false,
basicstyle=\small\ttfamily,
commentstyle=\color[named]{BrickRed}\ttfamily,
keywordstyle=\color[named]{Purple}\ttfamily,
%identifierstyle=\color[named]{Blue}\ttfamily,
%functionstyle=\color[named]{Blue}\ttfamily,
%typestyle=\color[named]{ForestGreen}\ttfamily,
stringstyle=\color[named]{Tan}\ttfamily,%
frame=none,%
backgroundcolor=\color{LightGrey}%
}

\lstdefinestyle{myPython}{%
%\lstset{%
language=python,
showstringspaces=false,
basicstyle=\small\ttfamily,
commentstyle=\color[named]{BrickRed}\ttfamily,
keywordstyle=\color[named]{Purple}\ttfamily,
%identifierstyle=\color[named]{Blue}\ttfamily,
%functionstyle=\color[named]{Blue}\ttfamily,
%typestyle=\color[named]{ForestGreen}\ttfamily,
stringstyle=\color[named]{Tan}\ttfamily,%
frame=none,%
backgroundcolor=\color{LightGrey}%
}
\end{latexonly}

%%%% I need the latexonly call here (even with the % signs, but
%%%% without the slash) as this
%%%% helps latex2html *not* see the stuff inbetween
%end{latexonly}

% put in an index?
\usepackage{makeidx}
\makeindex

