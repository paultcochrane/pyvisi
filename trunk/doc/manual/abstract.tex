% $Id$

\chapter*{Abstract}

\section*{About this manual}

This manual has been split into five parts in an attempt to cover all of the
material necessary to be able to use and master \pyvisi, but also to provide an
entry point for novices and experts alike.

Part~\ref{part:tutorial} is a very simple introduction to \pyvisi, and
discusses how to use \pyvisi to visualise ones data.  Novice users may wish to
start with \Chap{chap:tutFromScratch} (Starting from scratch) to help
themselves get going with \pyvisi.  

Part~\ref{part:userManual} gives user-level information about \pyvisi and
generic ideas behind the interface and how to use it (well, hopefully in the
future anyway).

Part~\ref{part:languageReference} gives specific information about the objects
and methods available for use in \pyvisi.  For renderer-specific documenation
see either the relevant section of this manual (to come) or the renderer
module's own documentation.

Part~\ref{part:developerManual} gives developer-level information on what
developers of renderer modules need to provide, and how they can do it.

Part~\ref{part:appendix} is an appendix and will cover any extra items of
interest, and includes the GNU General Public License, and a bibliography.

\section*{Tools used to build \pyvisi}

These are the multifarious tools with which \pyvisi, its documentation (both
handmade and automatically generated) and its web pages, has been made.

\begin{itemize}
\item General development tools: cvs, aap, dia, pyscript, pylint
\item Editors: emacs, vim
\item Linux Distributions: Gentoo Linux, Fedora Linux
\item Scripting tools and languages: \htmladdnormallink{python}{http://www.python.org}~\cite{web:python}
\item Documentation tools: \LaTeX, latex2html, epydoc, doxygen
\item Organisations: sourceforge.net, ESSCC, ACcESS
\end{itemize}

\section*{Feedback}

Yes, we want feedback!  If you have any comments about \pyvisi and/or this
manual (such as, inaccuracies, possible improvements, new features, what it
does well, etc.) then please email one of the current developer or the \pyvisi
web page webmaster.  You can find the addresses of both of these people on the
\pyvisi web page: \htmladdnormallink{http://pyvisi.sourceforge.net}
{http://pyvisi.sourceforge.net}.  And please, feel free to mention anything, no
matter how small.  It would be great to see \pyvisi improve the way people
want, and for it to be documented the way the \pyvisi user community wants.
